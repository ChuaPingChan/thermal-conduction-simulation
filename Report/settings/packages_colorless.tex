%% packages

\usepackage{blindtext} % needed for creating dummy text passages
\usepackage{amsmath} % needed for command eqref
\usepackage{amssymb} % needed for math fonts

\usepackage{pifont}% For ticks and crosses as defined in the newcommands below. More info at http://ctan.org/pkg/pifont
\newcommand{\cmark}{\ding{51}}%
\newcommand{\xmark}{\ding{55}}%

%%%%%%%%%%%%%%%%%%%%%%%%%%%%%%%%%%%%%%%%%%%%%%%%%%%%%%%%%%%
\usepackage[
	colorlinks=false,	% color links
	breaklinks %in case some labels are too long and may span several pages, e.g. in List of Figures, etc.
	]{hyperref} % needed for creating hyperlinks in the document, the option colorlinks=true gets rid of the awful boxes, breaklinks breaks long links (list of figures), and ngerman sets everything for german as default hyperlinks language
\usepackage[hyphenbreaks]{breakurl} % ben�tigt f�r das Brechen von URLs in Literaturreferenzen, hyphenbreaks auch bei links, die �ber eine Seite gehen (mit hyphenation).

\usepackage{xcolor}
\definecolor{c1}{rgb}{0,0,1} % blue
\definecolor{c2}{rgb}{0,0.3,0.9} % light blue
\definecolor{c3}{rgb}{0.3,0,0.9} % red blue
%\hypersetup{
%    linkcolor={c1}, % internal links
%    citecolor={c2}, % citations
%    urlcolor={c3} % external links/urls
%}

%\usepackage{cite} % needed for cite
\usepackage[round,authoryear]{natbib} % needed for cite and abbrvnat bibliography style
\usepackage[nottoc]{tocbibind} % needed for displaying table of contents, bibliography, index and other in the table of contents. 'nottoc' is to exclude the phreases "table of contents" from the table of contents.
%%%%%%%%%%%%%%%%%%%%%%%%%%%%%%%%%%%%%%%%%%%%%%%%%%%%%%%%%%%

\usepackage{graphicx} % needed for \includegraphics
\usepackage{subcaption} % for creating subfigures (side-by-side) and captioning them
\usepackage{float}	% to insert tables at precise position of code with [H] specifier.
\usepackage{longtable} % needed for long tables over pages
\usepackage{bigstrut} % needed for the command \bigstrut
\usepackage{enumerate} % needed for some options in enumerate
\usepackage{todonotes} % needed for todos
\usepackage{makeidx} % needed for creating an index
\makeindex
\usepackage{color}
\usepackage{siunitx} % for commonly used SI unit symbols with command SI{value}{unit}, SIrage, etc.
\usepackage{pdfpages} % for inserting pdf pages with command \includepdf{<filename>}
\usepackage{listings} % to insert source code of any programming language
\lstset{
basicstyle=\footnotesize,
frame=single,
numbers=left,
keepspaces=true,
breaklines=true,
%commentstyle=\color[rgb]{0,0.6,0},
%keywordstyle=\color{blue},
%stringstyle=\color[rgb]{0.8,0,0}
}